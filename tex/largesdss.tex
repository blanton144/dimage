\documentclass[10pt,preprint]{aastex}
\usepackage{psboxit}

\newcommand{\kms}{\,km~s$^{-1}$} 
\def\squig{\sim\!\!}
\newcommand{\Msun}{\mbox{\,$M_{\odot}$}}
\newcommand{\Lsun}{\mbox{\,$L_{\odot}$}}

\PScommands
\newcommand{\graybox}[1]{\psboxit{box 0.7 setgray fill}{\spbox{#1}}}

\newcommand{\mdmlimit}{0.5}
\newcommand{\vmaxmean}{56}
\newcommand{\mrmean}{-14.7}

\newcommand{\latin}[1]{{#1}}
\newcommand{\ie}{\latin{i.e.}}
\newcommand{\eg}{\latin{e.g.}}
\newcommand{\cf}{\latin{c.f.}}
\newcommand{\Sersic}{S\'ersic}
\newcommand{\vv}[1]{{\bf #1}}
\newcommand{\df}{\delta}
\newcommand{\dfft}{{\tilde{\delta}}}
\newcommand{\betaft}{{\tilde{\beta}}}
\newcommand{\erf}{{\mathrm{erf}}}
\newcommand{\erfc}{{\mathrm{erfc}}}
\newcommand{\Step}{{\mathrm{Step}}}
\newcommand{\ee}[1]{\times 10^{#1}}
\newcommand{\avg}[1]{{\langle{#1}\rangle}}
\newcommand{\Avg}[1]{{\left\langle{#1}\right\rangle}}
\def\simless{\mathbin{\lower 3pt\hbox
	{$\,\rlap{\raise 5pt\hbox{$\char'074$}}\mathchar"7218\,$}}} % < or of order
\def\simgreat{\mathbin{\lower 3pt\hbox
	{$\,\rlap{\raise 5pt\hbox{$\char'076$}}\mathchar"7218\,$}}} % > or of order
\newcommand{\iras}{{\sl IRAS\/}}
\newcommand{\petroratio}{{{\mathcal{R}}_P}}
\newcommand{\petroradius}{{{r}_P}}
\newcommand{\petronumber}{{{N}_P}}	
\newcommand{\petroratiolim}{{{\mathcal{R}}_{P,\mathrm{lim}}}}
\newcommand{\band}[2]{\ensuremath{^{#1}\!{#2}}}
\newcommand{\Vmax}{\ensuremath{V_\mmax}}
\newcommand{\mmax}{\ensuremath{\mathrm{max}}}
\newcommand{\mmin}{\ensuremath{\mathrm{min}}}
\newcommand{\minmax}{\ensuremath{\mathrm{\left\{^{min}_{max}\right\}}}}
\newcommand{\fixMr}{\ensuremath{M_\fixr}}
\newcommand{\fixr}{\ensuremath{{r}}}
\newcommand{\fixredshift}{0.1}
\newcommand{\fixmag}[1]{\ensuremath{{^{\fixredshift}\!{#1}}}}

\setlength{\footnotesep}{9.6pt}

\newcounter{thefigs}
\newcommand{\fignum}{\arabic{thefigs}}

\newcounter{thetabs}
\newcommand{\tabnum}{\arabic{thetabs}}

\newcounter{address}

\shortauthors{Blanton {\it et al.} (2006)}
\shorttitle{Large galaxy photometry in SDSS}

\begin{document}

\title{ Robust sky subtraction and large galaxy photometry \\
in the Sloan Digital Sky Survey}

\author{
Michael R. Blanton\altaffilmark{\ref{NYU}}}

%\altaffiltext{1}{Based on observations obtained with the
%Sloan Digital Sky Survey\label{SDSS}} 
\setcounter{address}{1}
\altaffiltext{\theaddress}{
	\stepcounter{address}
	Center for Cosmology and Particle Physics, Department of Physics, New
	York University, 4 Washington Place, New
	York, NY 10003
	\label{NYU}}

\begin{abstract}
We describe a technique for creating properly background subtracted
images from the Sloan Digital Sky Survey (SDSS) suitable for
photometric measurements on large galaxies. Our technique relies on
two critical steps: (1) fitting in an appropriately constrained way
for the smooth variation of the sky level within each SDSS drift scan;
and (2) mosaicking together images to measure galaxies that span
multiple drift scans. We perform full-up photometric simulations of
our procedure to demonstrate that we find systematics less than {\bf
some level} as a function of size, flux or surface brightness. This
method removes systematics found in the standard SDSS reductions.
\end{abstract}

\section{ Introduction: the formation of dwarf galaxies}
\label{intro}

Dwarf galaxies dominate the number density of galaxies in the
Universe, occurring in abundance both as satellites of their luminous
counterparts and in the regions in between luminous galaxies.  If the
hierarchical cold dark matter scenario is correct, the halos
containing dwarf galaxies were the first to collapse, and larger
haloes later grew at least in part by accreting dwarfs.  Because of
their early formation redshift, surviving dwarfs are unlikely to have
had a recent major merger.  Because the predictions of cosmological
scenarios for dwarfs differ in these ways from those for luminous
galaxies, dwarf galaxies provide an interesting test case for those
scenarios.

Here we consider one simple question about dwarf galaxies: do they
ever stop forming stars, the way that luminous galaxies do, and if so
why? Interestingly, we can obtain a partial answer to this question
--- there is a relatively rare class of dwarfs (dE and cE galaxies)
that has ceased star-formation, almost certainly because of
interactions with a larger galaxy. We are able to reach this
conclusion because we have an objectively selected sample of dwarf
galaxies, with accompanying optical photometry and spectra from the
Sloan Digital Sky Survey (SDSS; \citealt{york00a}), ultraviolet
photometry from GALEX, and 21 cm observations from the Green Bank
Telescope (GBT) and Arecibo Observatory.

For determining luminosity distances and other derived parameters from
observations, we have assumed cosmological parameters $\Omega_0 =
0.3$, $\Omega_\Lambda = 0.7$, and $H_0 = 100$ $h$ km s$^{-1}$
Mpc$^{-1}$. Where absolutely necessary, we have used $h=0.7$;
otherwise, we have left the dependence on $h$ explicit. All magnitudes
in this paper are $K$-corrected to rest-frame bandpasses using the
method of \citet{blanton03b} and {\tt kcorrect} {\tt v4\_1\_4}, unless
otherwise specified.  Because of the small range of look-back times in
our sample (a maximum of around 700 Myr), we do not evolution-correct
any of our magnitudes.

\section{ Constructing isolated and dwarf satellite samples}
\label{data}

\subsection{The SDSS catalog}

We use a modified version of the SDSS spectroscopic catalog.
\citet{blanton04b} describe our sample, which is a subsample of the
New York University Value-Added Galaxy Catalog
\citep[NYU-VAGC;][]{blanton05a}. We have updated that catalog from
SDSS Data Release 2 to Data Release 4 \citep[DR4;][]{adelman06a}.  The
\citet{blanton04b} catalog represents a significant improvement over
na\"ively selecting galaxies from the SDSS catalog, which is not
optimized for nearby, low surface brightness galaxies.  Our catalog
extends down to the spectroscopic flux limit of $m_r \sim 17.8$ used
by the SDSS.  For each galaxy, the catalog provides the SDSS redshift,
emission line measurements, multi-band photometry, structural
measurements and environment estimates (for more catalog details see
\citealt{blanton04b}).  Distances are estimated based on a model of
the local velocity field \citep{willick97a}.  Distance errors have
been folded into error estimates of all distance-dependent quantities
such as absolute magnitude and HI mass.

\subsection{ A tracer catalog from SDSS and RC3}

We are interested in where each galaxy is with respect to its
environment.  We cannot rely on the SDSS alone for this determination,
for several reasons.  First, the angular distances between nearest
neighbor galaxies can be large for this nearby sample--- for example,
searching a 1\,Mpc region around a galaxy 30\,Mpc away corresponds to
2 degrees on the sky.  Many of our dwarf galaxies are on the SDSS
Southern stripes, which are only 2.5 degrees wide.  In addition,
because the SDSS reduction software is not optimized for large,
extended objects and fails to process them correctly, the SDSS catalog
does not contain many of the bright galaxies within 30\,Mpc.  Thus, to
calculate the environments of our dwarf galaxy sample, we need a
supplemental catalog that extends beyond the SDSS area and contains
the brightest galaxies.

Both of these considerations drive us to use the The Third Reference
Catalog of Galaxies (RC3; \citealt{devaucouleurs91a}), which is a
nearly complete catalog of nearby galaxies. To determine environments
for our dwarf galaxies, we must determine the distance of each to its
nearest ``luminous'' neighbor. In this context, we define galaxies as
luminous when $M_r - 5\log_{10} h < -19$ (corresponding to circular
velocities of $V_c > 140$\kms\ for galaxies on the Tully-Fisher
relationship; \citealt{blanton07b}).  From the $B$ and $V$ photometry
listed in RC3, we infer $M_r$ for each galaxy.  For galaxies which
have the relevant entries listed, we call galaxies luminous if $M_r -5
\log_{10} h< -19$. For galaxies which do not have the relevant
entries, but do have HI data listed, we call them luminous if $W_{20}
> 300$ km s$^{-1}$ (as described in
\S\,\ref{radio}, $W_{20}$ is twice the maximum circular velocity of
the HI gas).  Finally, there are some galaxies with neither HI data
nor optical photometry listed in RC3. For this small set, we extract
the ``magnitude'' from NED (which empirically is very similar to the
$B$ band RC3 magnitude for galaxies which have both) and apply an
offset $M_r = M_{\mathrm{NED}} - 1$.  We call these galaxies luminous
if $M_r - 5\log_{10}h<-19$.  Additionally, we update the coordinates
in RC3 using the NASA Extragalactic Database (NED) coordinates for
each of the catalog objects.  This set of bright galaxies is not
perfectly uniform, but is suitable for our purposes.

We combine the SDSS galaxies with $M_r - 5\log_{10} h< -19$ with the
RC3 luminous galaxy catalog (removing repeats between the two) to
create a ``tracer'' catalog, with 29,352 entries. We will use this
tracer catalog to determine galaxy environment.

\subsection{ Finding isolated host galaxies}

For our analysis, we want to compare isolated dwarf galaxies to dwarf
satellites of luminous galaxies, but to {\it exclude} dwarf galaxies
in larger systems. In order to do so, we first must identify isolated
host galaxies. For every ``tracer'' galaxy, we determine the nearest
neighbor distance to another ``tracer'' galaxy within $r_p< 2$
$h^{-1}$ Mpc projected distance and 400\kms\ in redshift.

We define isolated host galaxies as tracer galaxies with a projected
nearest neighbor distance of $r_p>1$ $h^{-1}$ Mpc. Our of the original
tracer catalog, 10,187 are isolated host galaxies.
To treat thee images simultaneously, we use a new tool we have been
developing, called {\tt dimage}. This tool incorporates some of the
techniques used by the SDSS image analysis pipeline ({\tt photo};
\citealt{lupton04a}) but designed to be more user-friendly and to
handle multi-resolution data. Our procedure consists of three steps:
point-spread function (PSF) measurement, object detection, and
deblending of galaxies and stars.

\subsubsection{PSF measurement}

Our first step is to consider the images in each band separately, and
detect stars in the images to estimate the PSF.  For this purpose, we
subtract a background estimated by median smoothing the images with a
square 80$\times$80 pixel filter. Next, we smooth these subtracted
images by a Gaussian with a standard deviation of $1$ pixel. We
identify ``stars'' as contiguous regions of pixels, exceeding a flux
of 15$\sigma$, that are smaller than 80 pixels in size. Our initial
PSF estimate comes from median stacking the normalized images of these
``stars.'' With this initial estimate, we evaluate how well it
explains each ``star'' and reject poor $\chi^2$ fits.  After several
iterations with increasingly stringent criteria, we have the ``basic''
PSF. 

Next, we break each image up into 64 subregions on an 8$\times$8 grid.
In each subregion with stars, we stack the deviations from the
(scaled) basic PSF. We then calculate the three largest principal
components of these deviations. Following the procedure of
\citet{lupton03a}, we fit a third-order polynomial in $x$ and $y$ to
the three coefficients in each subregion.

In the analysis that follows, we use both the ``basic'' PSF and the
``variable'' PSF.  The latter is simply the basic PSF plus the
deviations predicted by the polynomial fit across the field applied to
the principal components.

In cases (as is common in the near UV) where there are stars in four
or fewer subregions in the field, our procedure defaults to the
``basic'' PSF. In cases where there are more than four but fewer than
sixteen filled subregions, we reduce the order of the fit polynomial
to the square root of that number minus one. In the rarer cases
(though not in the far UV!) that there are no objects that pass our
original ``star'' criterion, we simply assume a Gaussian PSF with a
standard deviation of 1.5 pixels in what follows.

\subsubsection{Object detection}

For object detection, we search for connected pixels that are above
the noise level.  In particular, we smooth the original images with a
Gaussian of the same full-width half-maximum (FWHM) as the basic
PSF. We then mark as ``detected'' any pixels greater than $5\sigma$
above the mean noise in the image. We replace detected pixels by zero
and then rebin the image to a coarser resolution by a factor of
two. We repeat our procedure for finding ``detected'' pixels, and
iterate one more time.  After rebinning, we smooth by a
correspondingly expanded Gaussian.  By rebinning and smoothing twice,
we include lower surface brightness features in our detections.

We break up the resulting detected pixels into contiguous regions,
which we designate as ``parents.''

\subsubsection{Deblending}

The most complex step in the process is the ``deblending'' of parents
into their constituent ``child'' objects. Parents often consist of
several independent physical components, and deblending attempts to
divide the light appropriately among these components.

First, we detect all peaks in the image and use the variable PSF to
test whether they are likely to be stars. Our procedure is simple: we
subtract out the best fit star in each location, median smooth the
subtracted image with a box about the FWHM size, and then ask by what
factor the original image exceeds the subtracted one.  If that factor
is above 10, we count the object as a PSF. 

\subsection{ Isolated and satellite dwarf galaxies}

Finally, we search for dwarf galaxies whose nearest neighbor within
400\kms\ (as defined by the projected distance $r_p$) is itself
isolated.  In this context, we define a dwarf galaxy to have $M_r -
5\log_{10} h > -17$; in our full sample there are 4,713 such dwarf
galaxies. Of these, a little less than half have an isolated nearest
neighbor. This final sample of 2,087 dwarfs then contains only
isolated dwarfs and satellite dwarfs of isolated hosts.

We note here that our results depend very little on the condition that
the nearest neighbor is isolated, and even less on the particular
criterion for isolation.  We apply the constraint because it makes the
interpretation of the difference between the isolated and satellite
dwarf populations more straightforward, by eliminating the possibility
of group/cluster related effects and by making the ``host'' galaxy
unambiguous.

Figure \ref{satpos} shows the projected distribution on the sky of
dwarfs ($M_r - 5\log_{10}>-17$) around isolated host galaxies ($M_r -
5\log_{10}<-19$), out to 2 $h^{-1}$ Mpc and within 400 \kms. Of the
full sample of 2,087 dwarfs, 681 do not appear on this plot because
they are further than 2 $h^{-1}$ Mpc from the nearest host galaxy. As
expected, the dwarfs cluster around the host galaxy; the median
distance is about 1.2 $h^{-1}$ Mpc. We show the projected distribution
aligned with the major and minor axes of the host galaxy, based on the
SDSS exponential model fit or the listed RC3 position angle, depending
on which catalog the host is in. The larger points indicate dwarfs
whose host galaxy is flat, with $b/a<0.4$. A Holmberg-like effect such
as that described by \citet{bailyn07a} for more luminous satellite
galaxies is not obviously present, but with only 100 satellites within
500 $h^{-1}$ kpc (around the appropriate radius for those results), we
draw no particular conclusions.

\section{ Photometric measurements from SDSS and GALEX}

We are interested in understanding the star-formation rates and
stellar populations of these dwarf galaxies as a function of distance
from the host galaxy.  Broad-band colors of the galaxies can yield key
insights into these properties; for all of our galaxies we can use the
SDSS imaging. To obtain a better handle on the star-formation rates we
also consider GALEX imaging where available.

For SDSS, we create a large mosaic around each galaxy based on the
underlying images. 


{\bf GALEX official numbers vs. ours}

{\bf SDSS test rerun (new photo version?)}

{\bf TODO:}

{\bf - deprojection code }

{\bf - what fraction of close blue galaxies are real pairs?}

{\bf - properties vs host }

{\bf - remeasure fluxes }

{\bf - fit Sersic profiles }

{\bf - SDSS metallicities and SFRs }

{\bf - measure GALEX fluxes, get SFRs that way }

{\bf - stack SDSS spectra vs separation }

\citet{blanton05a} did some pretty neat stuff.

\section{Deprojecting the satellite distribution}

Assuming isotropy, the mean projected radial distribution of
satellites must be related to the mean three-dimensional radial
distribution of satellites.  In particular, if we suppose that each
host galaxy is surrounded by a ``delta-function'' shell of radius $r$,
we can evaluate (given the locations of the host galaxies and the
geometry of the survey), the ``Greens function response'' that we
observe in the $r_p$ distribution. In more detail, we take a


\acknowledgments

Partial support for this work was provided by NASA-06-GALEX06-0300 and
NSF-AST-0428465.  MG acknowledges support from a Plaskett Research
Fellowship at the Herzberg Institute of Astrophysics of the National
Research Council of Canada. 

Funding for the creation and distribution of the SDSS Archive has been
provided by the Alfred P. Sloan Foundation, the Participating
Institutions, the National Aeronautics and Space Administration, the
National Science Foundation, the U.S. Department of Energy, the
Japanese Monbukagakusho, and the Max Planck Society. The SDSS Web site
is http://www.sdss.org/.

The SDSS is managed by the Astrophysical Research Consortium (ARC) for
the Participating Institutions. The Participating Institutions are The
University of Chicago, Fermilab, the Institute for Advanced Study, the
Japan Participation Group, The Johns Hopkins University, the Korean
Scientist Group, Los Alamos National Laboratory, the
Max-Planck-Institute for Astronomy (MPIA), the Max-Planck-Institute
for Astrophysics (MPA), New Mexico State University, University of
Pittsburgh, University of Portsmouth, Princeton University, the United
States Naval Observatory, and the University of Washington.

The Galaxy Evolution Explorer (GALEX) is a NASA Small Explorer. The
mission was developed in cooperation with the Centre National d'Etudes
Spatiales of France and the Korean Ministry of Science and Technology.

\bibliographystyle{../../../nyu-astro/tex/apj}
\bibliography{../../../nyu-astro/tex/apj-jour,../../../nyu-astro/tex/ccpp}

\newpage

%\include{satellites_figures}

\newpage
%\include{satellites_tables}

\end{document}
