\clearpage

\setcounter{thefigs}{0}

\clearpage
\stepcounter{thefigs}
\begin{figure}
\figurenum{\fignum}
\epsscale{0.60}
\plotone{Sky-fpbin.ps}
\caption{\label{fig:rawrun} Data that our sky model is fit to, for
part of an SDSS drift scan run.  The vertical direction is the scan
direction ($y$), the horizontal direction is perpendicular to the scan
direction ($x$). This image is the 8$\times$8 binned flat-fielded SDSS
data. Areas where no objects were detected show the original data.
Areas where objects were detected are replaced by the background sky
estimate plus noise. Each of the six vertical stripes represents a
``camcol'' and the black areas in between are not covered by a
CCD. The units of the image are raw counts, which therefore reflect
the relative gains of the different CCDs. }
\end{figure}

\clearpage
\stepcounter{thefigs}
\begin{figure}
\figurenum{\fignum}
\epsscale{0.60}
\plotone{Sky-mask.ps} 
\caption{\label{fig:mask} Mask applied to the data, for the same area
of sky as shown in Figure \ref{fig:rawrun}. White areas contribute to
the fit, black areas do not.}
\end{figure}

\clearpage
\stepcounter{thefigs}
\begin{figure}
\figurenum{\fignum}
\epsscale{0.60}
\plotone{Sky-model.ps} 
\caption{\label{fig:model} Final model sky fit, for the same area of
sky as shown in Figure \ref{fig:rawrun}. This model represents an
evaluation of Equation \ref{skyspline}.}
\end{figure}

\clearpage
\stepcounter{thefigs}
\begin{figure}
\figurenum{\fignum}
\epsscale{0.60}
\plotone{Sky-subtracted.ps} 
\caption{\label{fig:resid} The residuals of the data from the model
(literally Figure \ref{fig:rawrun} minus Figure \ref{fig:model}).}
\end{figure}

\clearpage
\stepcounter{thefigs}
\begin{figure}
\figurenum{\fignum}
\epsscale{0.60}
\plotone{skyqa.ps} 
\caption{\label{fig:skyqa} Distribution of residuals in each band in
areas with no detected objects. In each panel, we list the standard
deviation $\sigma$ of the residuals in nanomaggies per arcsec$^{-2}$, and the
fraction of areas that are $>3\sigma$ outliers from the
distribution. The smooth line is a Gaussian of with standard deviation
$\sigma$ for comparison. }
\end{figure}

\clearpage
\stepcounter{thefigs}
\begin{figure}
\figurenum{\fignum}
\epsscale{0.70}
\plotone{example-fig.ps} 
\caption{\label{fig:examples} Some example images made with the
	methods described in this paper. Upper left: Coma Cluster; upper
	right: NGC 5395; middle left: Messier 5; middle right: Messier 101;
	lower left: NGC 4656; lower right: UGC 3974.}
\end{figure}

\clearpage
\stepcounter{thefigs}
\begin{figure}
\figurenum{\fignum}
\epsscale{1.00}
\plotone{smosaic-local-qastats-magdiff.ps}
\caption{\label{fig:sdss_qa_magdiff} Magnitude differences between
the 7.3 arcsec aperture flux measured from our mosaic images, and that
reported by the SDSS catalog, for each SDSS band. The lines are the
16\%, 50\% and 84\% quantiles of the differences. The grey scale is
proportional to the conditional distribution of the flux ratio on the
magnitude At bright magnitudes, the differences are small, indicating
that these images have retained their overall photometricity. In each
panel, we list the mean value and dispersion of the differences in the
indicated magnitude range. }
\end{figure}

\clearpage
\stepcounter{thefigs}
\begin{figure}
\figurenum{\fignum}
\epsscale{1.00}
\plotone{smosaic-local-qastats-scaled-diff.ps}
\caption{\label{fig:sdss_qa_scaled} Similar to Figure
\ref{fig:sdss_qa_magdiff}, but now showing flux differences scaled to
the SDSS aperture flux uncertainty. In the right-hand panels we show
the distribution of each set of differences.  The smooth Gaussian
curve is the expected normal distribution. The flux differences are
not far removed from the normal distribution, indicating that our
images are not much degraded relative to the original images.}
\end{figure}


\clearpage
\stepcounter{thefigs}
\begin{figure}
\figurenum{\fignum}
\epsscale{1.00}
\plotone{montage-local-qastats-magdiff.ps}
\caption{\label{fig:montage_qa_magdiff} Similar to Figure
\ref{fig:sdss_qa_magdiff}, but using Montage images instead of our
mosaics. There is an overall scale difference in all bands of about
2--3\% (not a particularly large error if it is uniform). }
\end{figure}

\clearpage
\stepcounter{thefigs}
\begin{figure}
\figurenum{\fignum}
\epsscale{1.00}
\plotone{montage-local-qastats-scaled-diff.ps}
\caption{\label{fig:montage_qa_scaled} Similar to Figure
\ref{fig:sdss_qa_scaled}, but using Montage images instead of our
mosaics. We account for the overall scale difference found in in each band
before making these plots. The distribution of differences is much
broader than the SDSS uncertainties, indicating that for point sources
Montage images are significantly degraded.}
\end{figure}

\clearpage
\stepcounter{thefigs}
\begin{figure}
\figurenum{\fignum}
\epsscale{1.00}
\plotone{dr8_offsets_fakedist.model.ps}
\caption{\label{fig:fakedist} Distribution of $r$-band measured
  half-light sizes and magnitudes for our fake galaxy sample (circles)
  and for real SDSS galaxies (greyscale and contours). The two
  distributions follow each other well, but our fake sample is roughly
  size-limited at $r_{50} \sim 5$ arcsec.}
\end{figure}

\clearpage
\stepcounter{thefigs}
\begin{figure}
\figurenum{\fignum}
\epsscale{1.00}
\plotone{sky_offsets_ronly.model.ps}
\caption{\label{fig:sky_offsets_ronly} Errors in half-light radii $r_{50}$
  (top panel) and magnitudes (bottom panel) in the $r$-band, as a
  function of true galaxy $r_{50}$. The open circles show standard
  SDSS catalog photometry from the DR8 version of the pipeline; the
  DR7 version is similar but slightly worse. The filled circles show
  photometry and deblending using the methods described in this paper.
  The open squares show the photometry of the fake galaxies assuming
  perfect deblending.  The thick lines are running medians for each
  distribution. The thin lines are the smooth fit given by Equation
  \ref{eq:sky_offsets_model}, with parameters listed in Table
  \ref{table:sky_offsets}.}
\end{figure}

\clearpage
\stepcounter{thefigs}
\begin{figure}
\figurenum{\fignum}
\epsscale{1.00}
\plotone{sky_offsets_ugiz.model.ps}
\caption{\label{fig:sky_offsets_ugiz} Similar to Figure
  \ref{fig:sky_offsets_ronly}, but showing errors in colors with respect to
  $r$, using the $u$, $g$, $i$ and $z$ bands.}
\end{figure}

\clearpage
\stepcounter{thefigs}
\begin{figure}
\figurenum{\fignum}
\epsscale{1.00}
\plotone{sky_offsets_vs_r50meas.model.ps}
\caption{\label{fig:sky_offsets_vs_r50meas} Similar to Figure
  \ref{fig:sky_offsets_ronly}, but showing errors $r$-band magnitude against
  the measured value of $r_{50}$ rather than the true value. The
  dotted line shows the functional form given by \citet{hyde09a}.}
\end{figure}

