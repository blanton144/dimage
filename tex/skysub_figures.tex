\clearpage

\setcounter{thefigs}{0}

\clearpage
\stepcounter{thefigs}
\begin{figure}
\figurenum{\fignum}
\epsscale{0.60}
\plotone{Sky-fpbin.ps}
\caption{\label{fig:rawrun} Data that our sky model is fit to, for
part of an SDSS drift scan run.  The vertical direction is the scan
direction ($y$), the horizontal direction is perpendicular to the scan
direction ($x$). This image is the 8$\times$8 binned flat-fielded SDSS
data. Areas where no objects were detected show the original data.
Areas where objects were detected are replaced by the background sky
estimate plus noise. Each of the six vertical stripes represents a
``camcol'' and the black areas in between are not covered by a
CCD. The units of the image are raw counts, which therefore reflect
the relative gains of the different CCDs. }
\end{figure}

\clearpage
\stepcounter{thefigs}
\begin{figure}
\figurenum{\fignum}
\epsscale{0.60}
\plotone{Sky-mask.ps} 
\caption{\label{fig:mask} Mask applied to the data, for the same area
of sky as shown in Figure \ref{fig:rawrun}. White areas contribute to
the fit, black areas do not.}
\end{figure}

\clearpage
\stepcounter{thefigs}
\begin{figure}
\figurenum{\fignum}
\epsscale{0.60}
\plotone{Sky-model.ps} 
\caption{\label{fig:model} Final model sky fit, for the same area of
sky as shown in Figure \ref{fig:rawrun}. This model represents an
evaluation of Equation \ref{skyspline}.}
\end{figure}

\clearpage
\stepcounter{thefigs}
\begin{figure}
\figurenum{\fignum}
\epsscale{0.60}
\plotone{Sky-subtracted.ps} 
\caption{\label{fig:resid} The residuals of the data from the model
(literally Figure \ref{fig:rawrun} minus Figure \ref{fig:model}).}
\end{figure}

\clearpage
\stepcounter{thefigs}
\begin{figure}
\figurenum{\fignum}
\epsscale{1.00}
%\plotone{mosaic-examples.ps} 
\caption{\label{fig:examples} {\bf make some examples here}}
\end{figure}

\clearpage
\stepcounter{thefigs}
\begin{figure}
\figurenum{\fignum}
\epsscale{1.00}
\plotone{J110246.60+114710.6-qastats-scaled-diff.ps}
\caption{\label{fig:sdss_qa_magdiff} Magnitude differences between
the 7.3 arcsec aperture flux measured from our mosaic images, and that
reported by the SDSS catalog, for each SDSS band. The thin solid lines
are the expected 1$\sigma$ uncertainties. The thick red lines are the
16\%, 50\% and 84\% quantiles of the differences. At bright
magnitudes, the differences are small, indicating that these images
have retained their overall photometricity. However, at faint
magnitudes our aperture fluxes are overestimates relative to the SDSS
magnitudes, almost certainly due to the fact that we are not
performing any local sky subtraction. {\bf should we be for this
plot??} }
\end{figure}

\clearpage
\stepcounter{thefigs}
\begin{figure}
\figurenum{\fignum}
\epsscale{1.00}
\plotone{J110246.60+114710.6-qastats-scaled-diff.ps}
\caption{\label{fig:sdss_qa_scaled} Similar to Figure
\ref{fig:sdss_qa_magdiff}, but now showing flux differences scaled to
the SDSS aperture flux uncertainty. In the right-hand panels we show
the distribution of each set of differences.  The smooth Gaussian
curve is the expected normal distribution. The flux differences are
not far removed from the normal distribution, indicating that our
images are not much degraded relative to the original images.}
\end{figure}


\clearpage
\stepcounter{thefigs}
\begin{figure}
\figurenum{\fignum}
\epsscale{1.00}
\plotone{J115852.49+051634.6-qastats-magdiff.ps} 
\caption{\label{fig:montage_qa_magdiff} Similar to Figure
\ref{fig:sdss_qa_magdiff}, but using Montage images instead of our
mosaics. There is an overall scale difference in all bands of about
2--3\% (not a particularly large error if it is uniform). }
\end{figure}

\clearpage
\stepcounter{thefigs}
\begin{figure}
\figurenum{\fignum}
\epsscale{1.00}
\plotone{J115852.49+051634.6-qastats-scaled-diff.ps} 
\caption{\label{fig:montage_qa_scaled} Similar to Figure
\ref{fig:sdss_qa_scaled}, but using Montage images instead of our
mosaics. We account for the overall scale difference in each band
before making these plots. The distribution of differences is much
broader than the SDSS uncertainties, indicating that for point sources
Montage images are significantly degraded.}
\end{figure}


