\documentclass[10pt,preprint]{aastex}
\usepackage{psboxit}

\newcommand{\kms}{\,km~s$^{-1}$}
\def\squig{\sim\!\!}
\newcommand{\Msun}{\mbox{\,$M_{\odot}$}}
\newcommand{\Lsun}{\mbox{\,$L_{\odot}$}}

\PScommands
\newcommand{\graybox}[1]{\psboxit{box 0.7 setgray fill}{\spbox{#1}}}

\newcommand{\mdmlimit}{0.5}
\newcommand{\vmaxmean}{56}
\newcommand{\mrmean}{-14.7}

\newcommand{\latin}[1]{{#1}}
\newcommand{\ie}{\latin{i.e.}}
\newcommand{\eg}{\latin{e.g.}}
\newcommand{\cf}{\latin{c.f.}}
\newcommand{\Sersic}{S\'ersic}
\newcommand{\vv}[1]{{\bf #1}}
\newcommand{\df}{\delta}
\newcommand{\dfft}{{\tilde{\delta}}}
\newcommand{\betaft}{{\tilde{\beta}}}
\newcommand{\erf}{{\mathrm{erf}}}
\newcommand{\erfc}{{\mathrm{erfc}}}
\newcommand{\Step}{{\mathrm{Step}}}
\newcommand{\ee}[1]{\times 10^{#1}}
\newcommand{\avg}[1]{{\langle{#1}\rangle}}
\newcommand{\Avg}[1]{{\left\langle{#1}\right\rangle}}
\def\simless{\mathbin{\lower 3pt\hbox
    {$\,\rlap{\raise 5pt\hbox{$\char'074$}}\mathchar"7218\,$}}} % < or of order
\def\simgreat{\mathbin{\lower 3pt\hbox
    {$\,\rlap{\raise 5pt\hbox{$\char'076$}}\mathchar"7218\,$}}} % > or of order
\newcommand{\iras}{{\sl IRAS\/}}
\newcommand{\petroratio}{{{\mathcal{R}}_P}}
\newcommand{\petroradius}{{{r}_P}}
\newcommand{\petronumber}{{{N}_P}}
\newcommand{\petroratiolim}{{{\mathcal{R}}_{P,\mathrm{lim}}}}
\newcommand{\band}[2]{\ensuremath{^{#1}\!{#2}}}
\newcommand{\Vmax}{\ensuremath{V_\mmax}}
\newcommand{\mmax}{\ensuremath{\mathrm{max}}}
\newcommand{\mmin}{\ensuremath{\mathrm{min}}}
\newcommand{\minmax}{\ensuremath{\mathrm{\left\{^{min}_{max}\right\}}}}
\newcommand{\fixMr}{\ensuremath{M_\fixr}}
\newcommand{\fixr}{\ensuremath{{r}}}
\newcommand{\fixredshift}{0.1}
\newcommand{\fixmag}[1]{\ensuremath{{^{\fixredshift}\!{#1}}}}

\setlength{\footnotesep}{9.6pt}

\newcounter{thefigs}
\newcommand{\fignum}{\arabic{thefigs}}

\newcounter{thetabs}
\newcommand{\tabnum}{\arabic{thetabs}}

\newcounter{address}

\shortauthors{Blanton {\it et al.} (2006)}
\shorttitle{Improved background subtraction for SDSS}

\begin{document}

\title{Improved background subtraction for the Sloan Digital Sky Survey}


\author{
Michael R. Blanton\altaffilmark{\ref{NYU}},
Eyal Kazin\altaffilmark{\ref{NYU}},
Benjamin A. Weaver\altaffilmark{\ref{NYU}}
}

%\altaffiltext{1}{Based on observations obtained with the
%Sloan Digital Sky Survey\label{SDSS}}
\setcounter{address}{1}
\altaffiltext{\theaddress}{
    \stepcounter{address}
    Center for Cosmology and Particle Physics, Department of Physics, New
    York University, 4 Washington Place, New
    York, NY 10003
    \label{NYU}}

\begin{abstract}
We describe a procedure for background-subtracting Sloan Digital Sky
Survey (SDSS) imaging in a manner that improves the resulting
detection and photometry of large galaxies on the sky. Our method
treats each drift scan run as a whole and fits a smooth function to
variation of the sky background to a heavily masked and binned image
of the run. We have run our procedure on the full Data Release 7 for
the Northern Galactic Cap. In this region, we have generated 1~deg by
1~deg mosaics for general distribution. As a test of quality, we
compare the photometry of point sources in these mosaics relative to
the SDSS catalog. {\bf and find?} More critically, we have also tested
the effect of our background subtraction on the photometry of large
galaxies by inserting fake galaxies into the raw data and attempting
to measure them after background subtraction. {\bf and find?} These
techniques and these mosaics will provide the basis of upcoming
studies of the local Universe.
\end{abstract}

\section{ To-do list}

{\bf mrb: run montage\_image on a few patches with large galaxies and
compare to us}

{\bf mrb: create residual qa test for each run}

{\bf eak: simple comparison of fake nearby to us}

{\bf eak: photo v5\_4 comparison of fake nearby to us}

{\bf eak: photo v5\_6 comparison of fake nearby to us}

\section{Why improve the sky-subtraction?}
\label{sec:intro}

Background subtraction of astronomical images is probably a formally
impossible task.  The ideal treatment of the data gathered by a
detector would be to explain (at reasonable $\chi^2$) the counts in
each pixel, using a physical model of the sky brightness and its
gradients, the telescopic optics and scattered light properties, the
astronomical and other sources of light, and the sensitivity, noise
and other properties of the detector.
%For such a model to be useful,
%it would have to successfully partition light among the individual
%astronomical sources as well, so that they could be photometered
%individually.
However, this task is impractical and likely intractable as well,
given the detail needed in such a model, the time variability of
conditions and instruments, and the probable necessity of treating all
the data simultaneously. For this reason, most practical applications
of background subtraction tend to focus on simple approximations that
are tractable as well as close to correct for the problems of
interest.

For the Sloan Digital Sky Survey (\citealt{york00a}), the largest
existing survey of the sky to date, the standard photometric pipeline
takes just such a practical and accurate approach ({\tt photo};
\citealt{lupton01a}). In the version released with Data Release 7
(DR7; \citealt{abazajian09a}), calculates the median-smoothed
background on a scale of $100\times100$ arcsec and subtracts that from
the image before object detection and measurement. For faint point
source photometry in this data set --- that is, not a
confusion-limited one --- this approach is highly accurate. In almost
all cases, any diffuse light is subtracted away (whatever its source)
while the point source itself is left untouched. For these sources, it
is not necessary to fully model the background --- just to separate
point sources from any diffuse sources. The resulting fluxes of stars
are generally repeatable at well below the percent level; that is, as
well as can be expected given the vagaries of determining the overall
calibration of the data set (\citealt{padmanabhan07a}).

However, for galaxies the approach that {\tt photo} uses is more
troublesome.  While galaxies of small enough angular extent are
treated similarly to the stars, larger galaxies can be substantially
affected. For typical galaxies this method of sky subtraction causes
an underestimate in their flux, size, and concentration
(\citealt{blanton05b}). This problem becomes particularly evident for the
nearby brightest cluster galaxies (\citealt{bernardi07a,
lauer07a}).

In \S\ref{sec:model} below we present a different method for sky
subtraction that is more accurate for large galaxies (while retaining
most of the accuracy for point sources). Our method begins with the
estimate of the sky background from a processed SDSS imaging run.  It
builds masks around all bright detected sources, and any known sources
close to but outside the edge of the imaging run. Then it fits a
smooth spline to the unmasked data, applying appropriate constraints
to regularize the problem in the presence of the heavy masking.

In \S\ref{sec:mosaic} we describe how we take sky subtracted SDSS
fields and mosaic them into single image. We have produced 1~deg by
1~deg FITS format mosaics over the entire Northern Galactic Cap
covered by the SDSS, and provide access to snap-shots up this area on
the web.

In \S\ref{sec:endtoend}, we test the resulting photometry of these images
in two ways.  First, we compare aperture photometry of point sources
to the standard SDSS aperture photometry, to quantify how much
degradation our mosaicking procedure and sky subtraction have
introduced.

Second, we insert bright galaxies into the raw data and reanalyze the
data completely.  We then compare our background-subtracted mosaicked
images to the original fake data to evaluate the fidelity of our
procedure. We also evaluate the performance of two versions of the
standard SDSS pipelines (one that corresponds to the SDSS DR7 and
another that is intended for DR8).

The results described here are available for the Northern Galactic Cap
portion of the SDSS imaging using a simple on-line
interface.\footnote{\tt http://sdss.physics.nyu.edu/mosaic/}. 
The only other publicly available mosaicking and background
subtraction facility for SDSS that we are aware of is the results of
the Montage package distributed by IRSA (\citealt{berriman03a}). We
compare their methodology to ours in \S\ref{sec:mosaic}, concluding
that their background subtraction procedure in principle could be
superior to ours, but that their ``drizzle'' approximation to
interpolation is likely to cause errors for PSF photometry. In
\S\ref{sec:psfs}, we demonstrate that the PSF photometry is indeed
degraded relative to the original SDSS photometry for Montage
mosaics. In \S\ref{sec:simplegal}, we compare our bright galaxy
photometry to that from Montage images, {\bf finding what?}  As we
describe below, these problems are likely not fundamental to the
Montage background subtraction method.

\section{A model fit to the SDSS sky background}
\label{sec:model}

\subsection{The SDSS imaging data}
\label{sec:data}

The SDSS has taken $ugriz$ CCD imaging of $>10^4~\mathrm{deg^2}$ of
the sky \citep{york00a,abazajian09a}.  Automated software performs all
of the data processing: astrometry \citep{pier03a}; source
identification, deblending and photometry \citep{lupton01a};
photometricity determination \citep{hogg01a}; and calibration
\citep{fukugita96a,smith02a, padmanabhan07b}.

As \citet{gunn05a} describe, the SDSS focal plane has 30 imaging CCDs
evenly spaced (with gaps) across it, six ``camcols'' of each filter
($u$, $g$, $r$, $i$ and $z$). Each exposure is taken in drift scan
mode, such that a point in the sky passes through each CCD
sequentially with a gap of a minute or so.  The native pixel scale on
the CCDs is approximately 0.396 arcsec; when we refer to pixels in
this paper it is this native scale we will mean. When we discuss these
images, we will refer to the position in the scan direction as the
``pixel row'' or $y$, and the position perpendicular to the scan
direction as the ``pixel column'' or $x$.

Figure \ref{fig:rawrun} shows this geometry for part of run number
1336 in the $r$-band, showing each camcol horizontally and the gaps in
between each one. Thus each such run can be considered to consist of
thirty long rectangular images, one in each filter and camcol. In the
standard SDSS pipeline, each camcol is further broken into multiple
``fields,'' each with a width of 1361 pixels in the scan direction.

The image shown is approximately the standard SDSS pipeline's estimate
of the background light used for photometry, binned 8$\times$8. It is
equal to the actual counts in areas where there were no detected
objects, and is equal to the background estimate plus appropriate
noise in areas where there were detected objects. This determination
is made before any calibration except for the flat-field correction
--- which is only a function of pixel column since the observations
are drift scans.  It is about 2.3 deg wide and represents about 25
minutes of exposure time. The units are raw counts --- because of the
varying gains among the CCDs two of camcols mostly saturate in this
particular stretch.

SDSS's background light estimation consists of a median-smoothed image
on a scale of 100$\times$100 arcsec. Clearly Figure \ref{fig:rawrun}
demonstrates that bright objects have a substantial amount of their
light removed during background subtraction --- the background shows
clearly the presence of bright stars.  We will proceed by fitting a
much smoother function to this SDSS sky estimate in order to model the
clear variation with the background with time but not oversubtract the
brightest objects.

\subsection{Masking the data}
\label{sec:mask}

The first step is to mask the data appropriately. Clearly whatever
smooth function we use, we cannot allow it to be strongly affected by
the presence of bright stars. We begin by defining an initial mask
around known bright objects.

First, we identify any objects detected in the SDSS catalog with
$m<15$ in the filter in question. We create a mask of size
32$\times$32 native SDSS pixels around the faintest such objects,
growing with decreasing magnitude to 1600$\times$1600 pixels at $m=12$
(and constant with magnitude for brighter objects).  This mask handles
stars inside the imaging run well.

However, galaxies within the imaging run are frequently shredded into
fragments by the SDSS deblender, and it is more complex to determine
what area to mask based on the SDSS catalog alone. Therefore, we also
identify any galaxies from the Third Reference Catalog (RC3;
\citealt{devaucouleurs91a}) that land within the imaging run and build
masks around those objects that are 1000$\times$1000 pixels.

Finally, the SDSS filters suffer from internal reflection at their
edges, causing stars just outside the frame to scatter light onto the
CCD. Our model is too smooth to subtract these internal reflections,
and we do not want its large scale features affected by these local
phenomena.  Therefore, we mask these reflections by identifying any
stars from the Tycho-2 catalog that lie just outside the frame and
masking a rectangular region 1500$\times$900 pixels centered on the
star (the reflection has a larger extent in the pixel column direction
than in the row direction).

Occasionally SDSS fields have such a bright object inside of them that
the SDSS pipeline fails to reduce them at all.  We mask the entirety
of such fields from our fits.

Figure \ref{fig:mask} shows this initial mask for the run shown in
Figure \ref{fig:rawrun}. White areas are those to be used in the fit,
while black areas are those to be ignored. The mask is intentionally
very conservative --- we want to minimize the flux that is incorrectly
assigned to the sky background. Furthermore, as described in the next
section, we apply further iterative masking based on the results of
our fit.

\subsection{Spline model fit}
\label{sec:spline}

In order to fit the data, we first further bin it from 8$\times$8 to
8$\times$680 --- that is, we heavily bin it in the pixel row
direction.  We keep the resolution in the pixel column direction
because the sky ends up tracking some residual flat-fielding
errors. When rebinning, we account for the weights.

The spline model that we fit to the final binned data is as follows:
\begin{equation}
\label{skyspline}
    f(\mathrm{camcol}, x, y) = S(\mathrm{camcol}) \times
    Y(\mathrm{camcol}, y) \times X(\mathrm{camcol}, x, y),
\end{equation}
where $x$ and $y$ are the pixel column and row positions.
$S(\mathrm{camcol})$ is an overall scaling factor for each
camcol. $Y(\mathrm{camcol}, y)$ is a second-order b-spline for each
camcol, with break-points spaced approximately per SDSS field (1361
pixels). Thus, it expresses the overall variation of the sky over time
during the scan.  $X(\mathrm{camcol}, x, y)$ is a second-order
b-spline in two dimensions, with break-points in the $x$ direction
spaced once every 8 pixels and in the $y$ direction spaced every 40
fields. Thus, it allows for a rapid variation with column (which
accounts for flat-fielding errors) but allows that pattern to vary
slowly over time.

To fit the data, we (approximately) perform $\chi^2$ minimization of
the data with respect to the model. In detail, we first determine $S$
by taking a median across each camcol.

Second, we divide each camcol by $S$ and perform $\chi^2$ minimization
to fit $Y$ alone. We quadratically couple the spline parameters
strongly in the row direction to keep the fit smooth. In addition, we
allow a small quadratic coupling between the camcols in order to
interpolate more smoothly over masked data. These quadratic couplings
retain the linearity of the $\chi^2$ fit.

Third, we divide the data by $S$ and $Y$, and finally perform a second
$\chi^2$ minimization, this time for $X$. Again, we allow coupling
between the spline parameters.  In this case, we allow weak coupling
between the parameters in the row direction but very strong coupling
in the column direction, to keep $X$ as smooth as it can be. In the
$\chi^2$ minimizations for $Y$ and $X$, we use the mask weights, but
no inverse variance weighting.

We have a special condition for CCDs that have two amplifiers rather
than one. For such CCDs, half the pixels are read out by one device
and half by another; therefore, we do not couple the pixels across the
divide, and indeed we observe that the gains of the amplifiers do
drift at the fraction of the percent-level over the course of each
run.

Finally, given a fit function $f(\mathrm{camcol}, x, y)$ we estimate
an rms dispersion $\sigma$ around the fit, and then mask all regions
that are greater than 2$\sigma$ from the fit (in addition to the
initial mask described in \S\ref{sec:mask}). We then iterate the fit
four more times.

An example final fit is shown in Figure \ref{fig:model}. This model is
non-parametric and not physical --- consequently, for diffuse, but
real, sources many arcminutes in size, our method will by design
oversubtract the light.  However, Section
\S\ref{sec:simplegal} below shows that it is sufficiently good to
allow accurate measurements of most nearby galaxies.

Figure \ref{fig:resid} shows the residuals of the data after
subtracting the model sky.  There are remaining discrepancies visible,
which in the next subsection we will show to be at a level that is
fainter than 26 mag arcsec$^{-2}$.

\subsection{Residual tests}
\label{sec:residuals}

We have performed a simple residual test by considering random
unmasked patches 13$\times$13 native SDSS pixels in size. We choose
the areas that the SDSS photometric pipeline has identified as ``sky''
objects, without identified objects.  

Figure \ref{fig:skyqa} shows the distribution of the mean fluxes per
square arcsec in such patches across all SDSS runs. The performance in
most areas is excellent, with a standard deviation representing
residuals fainter than $26$ mag arcsec$^{-2}$ in the $r$-band. There
are a handful of outliers, which we find nearly always to be
associated with very extended emission from the interstellar medium or
an extremely bright star. While in regions of very extended emission,
we cannot guarantee that our sky estimate is correct, we do expect
that they will appear as bright outliers in this residual plot.

These residual tests are adequate insofar as they test the performance
of our fitting procedure and help to identify runs with highly
variable backgrounds (astronomical or instrumental).  However, they do
not provide a direct estimate of the effects of sky subtraction errors
on the resulting galaxy photometry --- that is, they do not test
whether we might be oversubtracting light near large sources.  We test
that effect more thoroughly in \S\ref{sec:simplegal}.

\section{Mosaicking the SDSS}
\label{sec:mosaic}

\subsection{Generating mosaics}

The point of our sky subtraction is to improve measurements of large
objects on the sky.  Many such objects overlap the edges of SDSS
fields (and some are bigger than or comparable to the size of a single
field).  Thus, to recover their photometry we must mosaic together
multiple fields after the background subtraction. Here we describe our
procedure for doing so.

For each desired mosaic we specify a desired World Coordinate System
header (WCS; \citealt{greisen02a}). We identify all SDSS fields that
overlap the desired area, and pick the minimal set of photometric
fields that cover that area.

Then we prepare each field for mosaicking.  Starting with the raw SDSS
data, we apply the flat field determined by the ubercalibration
procedure (\citealt{padmanabhan07b}). Using a procedure similar to
that used by the SDSS pipeline, we identify cosmic rays
(\citealt{lupton01a}). We interpolate over saturated pixels and cosmic
rays using simple linear interpolation in the $x$ direction.  Then we
estimate the sky by evaluating the function described by Equation
\ref{skyspline} at each pixel, and subtract it. In the sky estimate,
we account for any difference in the flat field as originally used by
the SDSS photometric pipeline, and as determined by
ubercalibration. Finally, we apply the photometric calibration using
the results of \citet{padmanabhan07b} for each field.

To resample each image we evaluate the position of each desired pixel
within each original field. Then we interpolate the original field's
image to the desired set of locations, using the well-known cubic
approximation to the sinc function (\citealt{park83a}). This
resampling method is known to work well for Nyquist-sampled images, a
condition SDSS virtually always satisfies. In practice, we use the
built-in IDL utilities {\tt polywarp} and {\tt poly\_2d} to perform
this resampling.

We evaluate the weighted average of the flux from all the input images
at each output pixel. The weights are unity throughout most of each
input image, but are apodized smoothly to zero at the edges. Thus, the
transition between regions which overlap two fields are relatively
smooth. Any output pixels that have no overlapping input images are
set to zero.

\subsection{Example mosaics}

Figure \ref{fig:examples} show several example mosaics created with
this procedure.  We have obviously chosen these for 

\subsection{Distribution of mosaics}

Although our procedures allow us to evaluate any arbitrary mosaic, for
ease of distribution we have created a set of 1$\times$1 deg mosaics
across the entire observed Northern Galactic Cap, or almost 8000
deg$^2$. We have created a web interface that allows users to easily
extract sections of these mosaics.\footnote{\tt
http://sdss.physics.nyu.edu/mosaic/}

This interface can be used either interactively or
non-interactively. The interactive method allows the user to browse a
Google sky viewer and select the region of interest.  The
non-interactive method is explained in the documentation.  In either
case, the user is returned FITS format images in each desired
bandpass, with correct WCS headers, sky-subtracted and calibrated. The
units of the images are nanomaggies per pixel.

\subsection{Comparison to Montage}

The only other publicly available mosaicking and background
subtraction facility for SDSS that we are aware of is the results of
the Montage package distributed by IRSA
\citep{berriman03a,berriman04a}. We will describe the quantitative
differences between their results and ours below.  Here we discuss
methodological differences.

The first major difference is in their method for background
subtraction. While in our case we rely on a smooth fit to regions with
no objects, Montage fits a smooth additive term within each run to
minimize the differences between it and other overlapping images. In
regions with many overlapping images, this procedure can be far
superior to ours. Over much of the SDSS area the only overlapping
images are at the north and south edges of each field, and how the
Montage algorithm behaves in that regime is a quantitative
question. We compare our results to theirs for large galaxies in
\S\ref{sec:simplegal} below.

The second major difference is in their method for ``resampling'' of
images. They do not perform a normal resampling but instead use a
generalized version of the ``drizzle'' algorithm. This algorithm
computes the fraction of each original pixel that overlaps each output
pixel, and distributes the original pixel's flux accordingly. Relative
to properly resampling a Nyquist-sampled image, such methods do a poor
job at maintaining the original PSF (or even at producing an image
with a well-defined PSF). {\bf refs??} They are also clearly
irreversible operations even in the noiseless case, demonstrating that
information is lost and the image is consequently degraded. We compare
our results to the Montage results for point sources in
\S\ref{sec:psfs} below, and believe that Montage's less good
at the bright end in those tests is due to their drizzling
procedure. However, it is unlikely that these issues affect our main
interest here, which is large galaxies on the sky.

\section{End-to-end tests of the sky background}
\label{sec:endtoend}

\subsection{Evaluating photometry}
\label{sec:e2eintro}

The residual tests of \S\ref{sec:residuals} are fine as far as they go,
but are of little use in evaluating the quality of the photometry of
objects in the masked regions.  In order to do that, we need to
compare to ground truth (possible to some extent for point sources;
\S\ref{sec:psfs}) or simulate the analysis of objects similar to
those we are interested in (necessary for galaxies;
\S\ref{sec:simplegal} and \S\ref{sec:deblendgal}).

\subsection{PSF photometry}
\label{sec:psfs}

The standard SDSS analysis should be at its finest for point sources.
In fact, because that analysis avoids interpolation, regridding or
stacking of the data we expect that its performance will be superior
than anything that our background-subtracted mosaics could
produce. Therefore, as a quality assurance tool to demonstrate how
photometric our images are, we compare aperture photometry of point
sources in our mosaics to that reported by the standard SDSS
pipeline. The main goal here is to verify that the overall scale of
the mosaicked images is correct and that the signal-to-noise has not
been degraded.

We randomly choose twenty 1 degree square mosaics. Because we are here
interested in point source photometry, before analysis we perform a
local sky subtraction, on the same scales as SDSS uses (100 arcsec
median smoothed).  We of course will not do so in the analysis of
large galaxies. From the overlapping SDSS catalog, we choose
unsaturated stars with no neighbors within 15 arcsec.  For each we
measure a circular aperture flux within a 7.3 arcsec radius (the
standard aperture used by the ubercalibration pipline).  Then, we
compare this flux to that reported within the same aperture by the
standard SDSS pipeline.

Figure \ref{fig:sdss_qa_magdiff} shows (as a function of magnitude)
the flux ratios between the 7.3 arcsec aperture flux measured from a
random set of twenty of our mosaic images, and that reported by the
SDSS catalog, for each SDSS band. The overall scale of the magnitudes
is close to correct, with less than a 1\% difference between our
results and SDSS at bright magnitudes.

Figure \ref{fig:sdss_qa_scaled} shows the distribution of flux
differences, scaled to the expected error distribution. The lines are
the 16\%, 50\% and 84\% quantiles. This distribution is close to the
normal distribution, an indication that our mosaicking procedure does
not degrade the image very badly.  Of course it does degrade to some
degree, an effect particularly noticeable at the faintest magnitudes,
where the median difference becomes almost 0.5$\sigma$ and the
distribution is significantly wider then the normal distribution.

We perform the same pair of tests on twenty Montage mosaics of the
same regions; results are shown in Figures
\ref{fig:montage_qa_magdiff} and \ref{fig:montage_qa_scaled}.  The
flux ratios reveal a scale difference between the Montage
images and the SDSS catalog of 3--5\%.
%This difference is suspiciously
%close to the fractional difference between the solid angle of a native
%SDSS pixel (0.396 arcsec) and the solid angle of the pixels output by
%Montage by default (0.4 arcsec).  The sense and magnitude of the error
%is similar to that which would result if the Montage images were
%output in units of flux per native SDSS pixel instead of flux per
%Montage-sized pixel. However,
We have not investigated their procedure thoroughly enough to diagnose
the cause of this error.  In any case, as an overall flux error this
problem is relatively minor for our purposes here.

In Figure \ref{fig:montage_qa_scaled}, we have first accounted for the
scale error before considering the flux differences. Having done so,
the error distribution clusters around zero.  At faint magnitudes the
distribution is similar to ours; at bright magnitudes it is
considerably broader. We suspect that this broad distribution is due
to the degradation of the image caused by the drizzling procedure. As
we see below, however, these errors appear to be less important when
considering larger, brighter objects. {\bf make sure this is true}

\subsection{Galaxy photometry: isolating the effect of the background}
\label{sec:simplegal}

In the previous section, we considered the effects of our procedures
on point source photometry, to verify that our images did reasonably
well relative to the ``ground truth'' established by the SDSS standard
photometry.  In the case of galaxy photometry, we do not trust the
SDSS standard results --- in fact, it is our goal to improve that
photometry substantially.  Therefore, we must use a different tactic
to test our results.

Our procedure is to insert fake galaxies into the raw data, using the
same tools developed by \citet{blanton05b} and used by several other
investigators since (\citealt{blanton04b, mandelbaum06a, masjedi06a}).
We distribute fake galaxy images onto a random set of locations on the
sky covered by SDSS fields. For each observation of an object in an
SDSS field in each band, we convert the fake stamps to SDSS raw data
units, convolve with the estimated seeing from the SDSS photometric
pipeline, and add Poisson noise using the estimates of the gain. We
add the resulting image to the real SDSS raw data, including the tiny
effects of nonlinearity in the response and the less tiny flat-field
variation as a function of column on the chip. We then run the SDSS
photometric pipeline {\tt photo} as well as our sky subtraction
procedure. Then, around each fake galaxy that we insert, we create a
mosaic using the tools described above.

{\bf describe here what exactly the fake sample is}

{\bf plots of flux differences between fake and stamps}

{\bf another test: direct comparison to Montage for some bright
galaxies}

\subsection{Deblended galaxy photometry}
\label{sec:deblendgal}

{\bf Comparisons of mock data to dimage measurements}

{\bf Comparisons of mock data to photo v5.4, v5.6}

\section{ Summary}

{\bf basic model}

{\bf outline of tests}

{\bf better than Montage: but Montage method a good idea}

\acknowledgments

Funding for the creation and distribution of the SDSS Archive has been
provided by the Alfred P. Sloan Foundation, the Participating
Institutions, the National Aeronautics and Space Administration, the
National Science Foundation, the U.S. Department of Energy, the
Japanese Monbukagakusho, and the Max Planck Society. The SDSS Web site
is http://www.sdss.org/.

The SDSS is managed by the Astrophysical Research Consortium (ARC) for
the Participating Institutions. The Participating Institutions are The
University of Chicago, Fermilab, the Institute for Advanced Study, the
Japan Participation Group, The Johns Hopkins University, the Korean
Scientist Group, Los Alamos National Laboratory, the
Max-Planck-Institute for Astronomy (MPIA), the Max-Planck-Institute
for Astrophysics (MPA), New Mexico State University, University of
Pittsburgh, University of Portsmouth, Princeton University, the United
States Naval Observatory, and the University of Washington.

This research made use of Montage, funded by the National Aeronautics and
Space Administration's Earth Science Technology Office, Computation
Technologies Project, under Cooperative Agreement Number NCC5-626 between
NASA and the California Institute of Technology. Montage is maintained by
the NASA/IPAC Infrared Science Archive.

\bibliographystyle{apj}
\bibliography{apj-jour,ccpp}

\newpage
\clearpage

\setcounter{thefigs}{0}

\clearpage
\stepcounter{thefigs}
\begin{figure}
\figurenum{\fignum}
\epsscale{0.60}
\plotone{Sky-fpbin.ps}
\caption{\label{fig:rawrun} Data that our sky model is fit to, for
part of an SDSS drift scan run.  The vertical direction is the scan
direction ($y$), the horizontal direction is perpendicular to the scan
direction ($x$). This image is the 8$\times$8 binned flat-fielded SDSS
data. Areas where no objects were detected show the original data.
Areas where objects were detected are replaced by the background sky
estimate plus noise. Each of the six vertical stripes represents a
``camcol'' and the black areas in between are not covered by a
CCD. The units of the image are raw counts, which therefore reflect
the relative gains of the different CCDs. }
\end{figure}

\clearpage
\stepcounter{thefigs}
\begin{figure}
\figurenum{\fignum}
\epsscale{0.60}
\plotone{Sky-mask.ps} 
\caption{\label{fig:mask} Mask applied to the data, for the same area
of sky as shown in Figure \ref{fig:rawrun}. White areas contribute to
the fit, black areas do not.}
\end{figure}

\clearpage
\stepcounter{thefigs}
\begin{figure}
\figurenum{\fignum}
\epsscale{0.60}
\plotone{Sky-model.ps} 
\caption{\label{fig:model} Final model sky fit, for the same area of
sky as shown in Figure \ref{fig:rawrun}. This model represents an
evaluation of Equation \ref{skyspline}.}
\end{figure}

\clearpage
\stepcounter{thefigs}
\begin{figure}
\figurenum{\fignum}
\epsscale{0.60}
\plotone{Sky-subtracted.ps} 
\caption{\label{fig:resid} The residuals of the data from the model
(literally Figure \ref{fig:rawrun} minus Figure \ref{fig:model}).}
\end{figure}

\clearpage
\stepcounter{thefigs}
\begin{figure}
\figurenum{\fignum}
\epsscale{0.60}
\plotone{skyqa.ps} 
\caption{\label{fig:skyqa} Distribution of residuals in each band in
areas with no detected objects. In each panel, we list the standard
deviation $\sigma$ of the residuals in nanomaggies per arcsec$^{-2}$, and the
fraction of areas that are $>3\sigma$ outliers from the
distribution. The smooth line is a Gaussian of with standard deviation
$\sigma$ for comparison. }
\end{figure}

\clearpage
\stepcounter{thefigs}
\begin{figure}
\figurenum{\fignum}
\epsscale{0.70}
\plotone{example-fig.ps} 
\caption{\label{fig:examples} Some example images made with the
	methods described in this paper. Upper left: Coma Cluster; upper
	right: NGC 5395; middle left: Messier 5; middle right: Messier 101;
	lower left: NGC 4656; lower right: UGC 3974.}
\end{figure}

\clearpage
\stepcounter{thefigs}
\begin{figure}
\figurenum{\fignum}
\epsscale{1.00}
\plotone{smosaic-local-qastats-magdiff.ps}
\caption{\label{fig:sdss_qa_magdiff} Magnitude differences between
the 7.3 arcsec aperture flux measured from our mosaic images, and that
reported by the SDSS catalog, for each SDSS band. The lines are the
16\%, 50\% and 84\% quantiles of the differences. The grey scale is
proportional to the conditional distribution of the flux ratio on the
magnitude At bright magnitudes, the differences are small, indicating
that these images have retained their overall photometricity. In each
panel, we list the mean value and dispersion of the differences in the
indicated magnitude range. }
\end{figure}

\clearpage
\stepcounter{thefigs}
\begin{figure}
\figurenum{\fignum}
\epsscale{1.00}
\plotone{smosaic-local-qastats-scaled-diff.ps}
\caption{\label{fig:sdss_qa_scaled} Similar to Figure
\ref{fig:sdss_qa_magdiff}, but now showing flux differences scaled to
the SDSS aperture flux uncertainty. In the right-hand panels we show
the distribution of each set of differences.  The smooth Gaussian
curve is the expected normal distribution. The flux differences are
not far removed from the normal distribution, indicating that our
images are not much degraded relative to the original images.}
\end{figure}


\clearpage
\stepcounter{thefigs}
\begin{figure}
\figurenum{\fignum}
\epsscale{1.00}
\plotone{montage-local-qastats-magdiff.ps}
\caption{\label{fig:montage_qa_magdiff} Similar to Figure
\ref{fig:sdss_qa_magdiff}, but using Montage images instead of our
mosaics. There is an overall scale difference in all bands of about
2--3\% (not a particularly large error if it is uniform). }
\end{figure}

\clearpage
\stepcounter{thefigs}
\begin{figure}
\figurenum{\fignum}
\epsscale{1.00}
\plotone{montage-local-qastats-scaled-diff.ps}
\caption{\label{fig:montage_qa_scaled} Similar to Figure
\ref{fig:sdss_qa_scaled}, but using Montage images instead of our
mosaics. We account for the overall scale difference found in in each band
before making these plots. The distribution of differences is much
broader than the SDSS uncertainties, indicating that for point sources
Montage images are significantly degraded.}
\end{figure}

\clearpage
\stepcounter{thefigs}
\begin{figure}
\figurenum{\fignum}
\epsscale{1.00}
\plotone{r_offsets.ps}
\caption{\label{fig:r_offsets} }
\end{figure}



%\newpage
%\clearpage

\setcounter{thetabs}{0}

\clearpage
\stepcounter{thetabs}
\begin{deluxetable}{llllll}
\tablewidth{0pt}
\tablecolumns{6}
\tablenum{\tabnum}
\tablecaption{\label{table:sky_offsets} Fits to residual sizes and fluxes}
\tablehead{ Quantity & Sky version & $a_0$ & $a_1$ & $a_2$ &
  $\sigma_a$ }
\startdata
$r_{50}$ & \texttt{v5\_4} & $-0.151$ & $-0.211$ & $-0.262$ & $0.134$\cr
 & \texttt{v5\_6} & $-0.146$ & $-0.193$ & $-0.269$ & $0.129$\cr
 & global & $-0.036$ & $0.033$ & $-0.212$ & $0.085$\cr
$m_r$ & \texttt{v5\_4} & $0.168$ & $0.641$ & $0.704$ & $0.173$\cr
 & \texttt{v5\_6} & $0.101$ & $0.486$ & $0.861$ & $0.237$\cr
 & global & $0.079$ & $0.032$ & $-0.001$ & $0.163$\cr
$(u-r)$ & \texttt{v5\_4} & $0.028$ & $0.184$ & $-0.047$ & $0.147$\cr
 & \texttt{v5\_6} & $-0.008$ & $0.113$ & $0.006$ & $0.198$\cr
 & global & $0.062$ & $-0.009$ & $-0.001$ & $0.073$\cr
$(g-r)$ & \texttt{v5\_4} & $-0.008$ & $-0.007$ & $0.004$ & $0.045$\cr
 & \texttt{v5\_6} & $-0.013$ & $-0.015$ & $0.010$ & $0.066$\cr
 & global & $0.001$ & $0.002$ & $-0.005$ & $0.016$\cr
$(r-i)$ & \texttt{v5\_4} & $-0.017$ & $-0.025$ & $0.040$ & $0.056$\cr
 & \texttt{v5\_6} & $-0.018$ & $-0.036$ & $0.026$ & $0.074$\cr
 & global & $0.008$ & $0.002$ & $-0.003$ & $0.021$\cr
$(r-z)$ & \texttt{v5\_4} & $-0.054$ & $-0.129$ & $0.122$ & $0.142$\cr
 & \texttt{v5\_6} & $-0.051$ & $-0.156$ & $0.046$ & $0.165$\cr
 & global & $-0.011$ & $-0.001$ & $0.001$ & $0.053$\cr
$m_r$ (vs $r_{50}$ meas.) & \texttt{v5\_4} & $0.379$ & $0.959$ & $0.530$ & $0.325$\cr
 & \texttt{v5\_6} & $0.293$ & $0.727$ & $0.131$ & $0.290$\cr
 & global & $0.092$ & $0.040$ & $-0.098$ & $0.177$\cr

\enddata
\tablecomments{Parameters are defined in Equation
  \ref{eq:sky_offsets_model}. Magnitude offsets are in units of
  magnitudes.  Half-light radius offsets are in units of dex. Results
  labeled {\tt v5\_4} are appropriate for the SDSS DR7
  catalog. Results labeled {\tt v5\_6} are appropriate for SDSS DR8
  catalog. Results labeled ``global'' are appropriate for the
  sky-subtraction and deblending analysis described in this paper.}
\end{deluxetable}



\end{document}
